\chapter{Filter formalism}\label{filter}
Next we present a very general filter formalism, which
is useful when dealing with the compressible equations of fluid dynamics. We
collect the most important rules\footnote{These rules have already been used
implicitely by \citet{Canuto1997,Schmidt2006}, but our work is the first to 
summarize them explicitly.}, which are necessary to filter the compressible
equations of fluid dynamics. 

\section{Reynolds filter}
In general filtering means splitting some quantity $a$ in a mean value $\fil{a}$
generated by the filter procedure and some deviation $a'$ from the mean value.
If a filter fulfills the following relations,
\begin{align}
\fil{A + B} &= \fil{A}+\fil{B},\label{eq:rey1} \\
\fil{C} &= C,\ \text{if $C = $\ const.}, \label{eq:rey2}\\
\fil{\fil{A} B} &= \fil{A} \fil{B}, \label{eq:rey3}
\end{align}
it is called a Reynolds filter or Reynolds operator. From equation
\eqref{eq:rey2} and
(\ref{eq:rey3}) we see that for $B=C=const.$ 
\begin{align}
\fil{\fil{A} C} = \fil{A} C. \label{eq:rey4}
\end{align}
From this relation \eqref{eq:rey4} it follows for $C=1$
\begin{align}
\fil{\fil{A}}=\fil{A}. \label{eq:rey5}
\end{align}
From the last equation (\eqref{eq:rey5}) and \eqref{eq:rey3} we get
for $B=\fil{D}$
\begin{align}
\fil{\fil{A}\fil{D}} = \fil{A}\fil{\fil{D}}=\fil{A}\fil{D}.
\end{align}
If we split a quantity $a$ in a sum of some kind of mean value $\fil{a}$
(computed by a filter
that satisfies the Reynolds criteria \eqref{eq:rey1}-\eqref{eq:rey3}) and some
deviations $a'$
\begin{align}
a=\fil{a}+a'
\end{align}
it follows from equation \eqref{eq:rey5} for the mean of the deviations
\begin{align}
\fil{a'}=\fil{a-\fil{a}}=\fil{a}-\fil{\fil{a}}=\fil{a}-\fil{a}=0.
\label{eq:meandev}
\end{align}
One can show that the so called central moments of this quantity
($\fil{a' b'}$, $\fil{a' b' c'}$, $\fil{a' b' c' d'}$, $\ldots$)
can be expressed in terms of the classical moments  
($\fil{a b}$, $\fil{a b c}$, $\fil{a b c d}$,~$\ldots$)
like\footnote{See for example \citep{Monin1971}.}
\begin{align}
\fil{a' b'}=&\ \fil{a b}-\fil{a}\fil{b}\label{eq:mom2}\\
\begin{split}
\fil{a' b' c'}=&\ \fil{a b c} - \fil{a}\fil{b}\fil{c}\\
&-\fil{a}\fil{b' c'} - \fil{b}\fil{a' c'} - \fil{c}\fil{a' b'}
\end{split}
 \\
\begin{split}
\fil{a' b' c' d'}=&\ \fil{a b c d} - \fil{a}\fil{b}\fil{c}\fil{d}\\
&-\fil{a}\fil{b' c' d'} -\fil{b}\fil{a' c' d'}
-\fil{c}\fil{a' b' d'} -\fil{d}\fil{a' b' c'}\\
&-\fil{a}\fil{b}\fil{c' d'}-\fil{a}\fil{c}\fil{b' d'}-\fil{a}\fil{d}\fil{b'
c'}\\
&-\fil{b}\fil{c}\fil{a' d'}-\fil{b}\fil{d}\fil{a' c'}-\fil{c}\fil{d}\fil{a' b'}
\end{split}
\\
\fil{a' b' c' d' e'} =&\ \ldots \label{eq:mom5}
\end{align}
\section{Germano formalism}
\citet{Germano1992} postulates that the relations between moments and central
moments for non-Reynolds operators\footnote{Non-Reynolds operators do not
fulfill \ref{eq:meandev}, so their mean of the deviations is unequal zero.} are
of similar form as for Reynolds operators.
Therefore he introduces the so called generalized central moments $\tau(a,b), 
\tau(a,b,c),\ldots$ for non-Reynolds operators. These should 
fulfill in analogy to equation \eqref{eq:mom2}-\eqref{eq:mom5} the following
relations\footnote{It seems to be very difficult to prove these relations even
in case of very simple non-Reynolds operators.}
\begin{align}
\tau(a,b)=&\ \fil{a b}-\fil{a}\fil{b} \label{eq:germ2}\\
\begin{split}
\tau(a,b,c)=&\ \fil{a b c} - \fil{a}\fil{b}\fil{c}\\
&-\fil{a}\tau(b, c) - \fil{b}\tau(a, c) - \fil{c}\tau(a, b)\label{eq:germ3}
\end{split}
\\
\begin{split}
\tau(a,b,c,d)=&\ \fil{a b c d} - \fil{a}\fil{b}\fil{c}\fil{d}\\
&-\fil{a}\tau(b, c, d) -\fil{b}\tau(a, c, d)\\
&-\fil{c}\tau(a, b, d) -\fil{d}\tau(a, b, c)\\
&-\fil{a}\fil{b}\tau(c, d)-\fil{a}\fil{c}\tau(b, d)-\fil{a}\fil{d}\tau(b, c)\\
&-\fil{b}\fil{c}\tau(a, d)-\fil{b}\fil{d}\tau(a, c)-\fil{c}\fil{d}\tau(a, b)
\end{split}
\\
\fil{a' b' c' d' e'} =&\ \ldots \label{eq:germ5}
\end{align}
For the generalized central moments the following rules apply:
\begin{enumerate}
\item  They are symmetric in their arguments
\begin{align}
\tau(a,b) = \tau(b,a) ; \tau(a,b,c)=\tau(b,a,c) , \ldots 
\end{align}
\item The generalized central moment of a constant is zero
\begin{align}
\tau(a,c) = 0, \tau(a,b,c) = 0,\ \text{if $c =$ const.}
\end{align}
\item In case of a static (time independent) filter operator 
it permutes with the time derivative and the chain rule applies
\begin{align}
\pd{t}\tau(a,b)=\tau\lra{\pd{t}a,b}+\tau\lra{a,\pd{t}b}
\end{align}
\item If the filter operator is isotropic (independent of 
position in space) then it applies
\begin{align}
\pd{x_i}\tau(a,b)=\tau\lra{\pd{x_i}a,b}+\tau\lra{a,\pd{x_i}b}
\end{align}
\item Additionally the following relation can be proved
\begin{align}
\tau(a,Cb) &= C\cdot \tau(a,b),\ \text{if $C =$ const.} \\
\tau(a,b+c) &= \tau(a,b) + \tau(a,c) \\
\tau(a,bc) &= \tau(a,b,c) + \fil{b}\tau(a,c) + \fil{c}\tau(a,b) 
\end{align}
\end{enumerate}
\section{Favre-Germano formalism} \label{FGform}
In the case of compressible fluid dynamics, the moments appearing in the
filtered equations are one order higher than in non-compressible
fluid dynamics (eg. $\fil{\rho u_i u_j}$ instead of $\fil{u_i u_j}$).
If we would adopt the Germano relations \eqref{eq:germ2} to \eqref{eq:germ5}
in this case, we would get many terms which are difficult to interpret
physically.
But if we use density weighted
quantities\footnote{For a modern review of this procedure
see \citet{Veynante2002}.} similar to \citet{Favre1969} and develop relations in
analogy to the Germano relations for these density weighted
quantities, we can write the filtered compressible equations of fluid dynamics
in a much simpler way \citep{Canuto1997,Schmidt2006}.

We define density weighted quantities according to Favre like
\begin{align}
\fil{\rho a} = \fil{\rho} \hat{a} \Rightarrow \hat{a}=\frac{\fil{\rho
a}}{\fil{\rho}}.
\end{align}
In analogy to Germano we postulate the following relations:
\begin{align}
\hat{\tau}(a,b)=&\ \fil{\rho a b} - \fil{\rho} \hat{a} \hat {b}\\
\begin{split}
\hat{\tau}(a,b,c)=&\ \fil{\rho a b c} - \fil{\rho} \hat{a} \hat{b} \hat{c} \\
&-\hat{a}\hat{\tau}(b,c)-\hat{b}\hat{\tau}(a,c)-\hat{c}\hat{\tau}(a,b)
\end{split}
\\
\hat{\tau}(a,b,c,d)=&\ \ldots
\end{align}
For the quantities $\hat{\tau}(\ldots)$ the same rules apply as for the 
generalized central moments $\tau(\ldots)$ introduced by Germano:
\begin{enumerate}
\item $\hat{\tau}(a,b) = \hat{\tau}(b,a) ; \hat{\tau}(a,b,c)=\hat{\tau}(b,a,c) ,
\ldots$
\item $\hat{\tau}(a,c) = 0, \hat{\tau}(a,b,c) = 0$, if $c =$ const.
\item
$\pd{t}\hat{\tau}(a,b)=\hat{\tau}\lra{\pd{t}a,b}+\hat{\tau}\lra{a,\pd{t}b}$ 
for static filter.
\item
$\pd{x_i}\hat{\tau}(a,b)=\hat{\tau}\lra{\pd{x_i}a,b}+\hat{\tau}\lra{a,\pd{x_i}b}
$
for isotropic filter.
\item
\begin{enumerate} 
\item $\hat{\tau}(a,Cb) = C\cdot \hat{\tau}(a,b)$ , if $C =$ const.
\item $\hat{\tau}(a,b+c) = \hat{\tau}(a,b) + \hat{\tau}(a,c)$ 
\item $\hat{\tau}(a,bc) = \hat{\tau}(a,b,c) + \fil{b}\hat{\tau}(a,c) +
\fil{c}\hat{\tau}(a,b)$
\end{enumerate} 
\end{enumerate}
If we compare the Favre relations to the Germano relations we see:
\begin{align}
\text{Germano:}\ \fil{\rho a} =&\ \fil{\rho}\fil{a} + \tau(\rho,a)\\
\text{Favre:}\ \fil{\rho a}=&\ \fil{\rho}\hat{a}\\
\Rightarrow \hat{a}=&\ \fil{a} + \frac{\tau(\rho,a)}{\fil{\rho}}\\
\Rightarrow \hat{a}=&\ \fil{a},\ \text{if $\rho =$ const.}
\end{align}
This means in the case of a constant density, the formalism with Favre density
weighted quantities
is equivalent to the Germano formalism.
\footnote{This can also be proved for the higher moments.}
\section{Explicit filtering}\label{expfil}
The rules for filtering described in the last sections do not depend on an
explicit form of a filter procedure. However we now want to introduce
a commonly used representation of a filter procedure, namely the convolution
filter. Using a convolution filter the mean value $\fil{a}$ of some
quantity $a(x)$ is defined as 
\begin{align}
\fil{a(x)} = \iinf G(x-x') a(x') dx', 
\end{align}
where $G(x-x')$ is called the convolution kernel, and is associated with
some cutoff length $l_{\Delta}$. The deviation of the mean value is then
defined as
\begin{align}
a' = a(x) - \fil{a(x)} = a(x) - \iinf G(x-x') a(x') dx'.
\end{align}
The importance of the convolution filter stems from the fact that it can be
used to generalise discrete operators, e.g. we can write the well-known
second-order central difference formula for the derivative of a continuous
variable like\footnote{See \citet{Rogallo1984}.}
\begin{align*}
\frac{a(x+h)-a(x-h)}{2h}&=\td{x}\lra{\frac{1}{2h}\int_{x-h}^{x+h} a(x') dx'}\\
&= \td{x} \iinf G(x-x') a(x') dx'\\
&= \td{x} \fil{a}
\end{align*}
with 
\begin{align}
G(x-x') = 
\begin{cases}
\frac{1}{2h} & \text{if }\abs{x-x'} \leq h \\
0 & \text{otherwise}
\end{cases}.\label{eq:box}
\end{align}
The convolution kernel \eqref{eq:box} is also called a box or top-hat filter and
is most often used for performing explicit spatial scale separation. 
   