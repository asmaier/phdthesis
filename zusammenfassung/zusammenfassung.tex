\chapter*{Zusammenfassung}
Galaxienhaufen sind die größten, gravitativ gebundenen Strukturen im Universum.
Nach dem hierarchischen Modell der Strukturentstehung, wonach größere,
gravitativ gebundene Systeme aus kleineren Systemen entstehen, sind sie damit
auch die jüngsten Strukturen im Universum. Daher kann die genaue Verfolgung
ihrer Entwicklung, z.B. anhand ihrer Anzahldichte abhängig von der Zeit, zur
Messung wichtiger kosmologischer Parameter verwendet werden. 

Viele Eigenschaften von Galaxienhaufen können mit optischen Beobachtungen
bestimmt werden, die derzeit verlässlichsten Daten erhält man jedoch durch
Beobachtungen mit Röntgensatelliten wie z.B. CHANDRA und XMM-Newton. Aus diesen
Daten ergeben sich Korrelationen z.B. zwischen Leuchtkraft und Temperatur oder 
zwischen Masse und Temperatur für Galaxienhaufen. Nimmt man an, dass alleine 
Gravitation eine Rolle bei der Entstehung von Galaxienhaufen spielt, sollten
diese selbstähnlich sein, jedoch stimmen die experimentell gefundenen
Relationen nicht mit dieser Annahme überein. Eine physikalische Erklärung,
warum die Selbstähnlichkeit verletzt ist, kann die Existenz von turbulenten
Strömungen in Galaxienhaufen sein.

Die bisherigen Untersuchungen von Turbulenz in Galaxienhaufen waren jedoch
darauf beschränkt, alleine passiv nach Eigenschaften der Strömung in
Galaxienhaufen zu suchen, die auf Turbulenz hinweisen. Die aktive Rolle der
Turbulenz, d.h. den möglichen Einfluss von klein und kleinstskaligen
Geschwindigkeitsfluktuationen auf die Strukturentstehung, zu modellieren war
bisher nicht möglich. Ein Grund dafür ist, das die akzeptierte
Kolmogorov-Theorie der Turbulenz nur für inkompressible, homogene und isotrope
Strömungen angewendet werden kann, Strömungen in der Astrophysik jedoch meist
kompressibel, selbstgravitierend und anisotrop sind. Ein zweiter Grund ist,
dass die derzeitigen Modelle zur numerischen
Beschreibung von Turbulenz (sog. Grobstruktursimulationen) auf der Filterung der
fluiddynamischen Gleichungen bei einer bestimmten charakteristischen Längenskala
beruhen, was im offenen Widerspruch zu numerischen Methoden der
Astrophysik steht, welche mit adaptiven Gittern arbeiten, um die vielen
Phänomene auf unterschiedlichsten Längenskalen in astrophysikalischen
Umgebungen darzustellen.

Ziel dieser Arbeit war es daher, ein neues numerisches Modell zu
entwickeln, welches es ermöglicht Grobstruktursimulationen auch mit adaptiven
Gittercodes auszuführen, um Turbulenz über große Längenskalenbereiche konsistent
zu simulieren. Da das frei verfügbare Programm Enzo zur Simulation
kosmologischer Strömungen adaptive, in Blöcken organisierte Gitter und den
wenig-dissipativen PPM-Lösungsalgorithmus benutzt, sollte unser neues Modell in
diesen Code implementiert werden. 

Um Grobstruktursimulationen mit Enzo zu ermöglichen, implementierten wir eine
neue Erhaltungsgleichung für die turbulente Energie in das Programm und
koppelten sie an die bereits vorhandenen Erhaltungsgleichungen von Impuls,
kinetischer und interner Energie. Wichtigster Punkt und zentrale Idee um
Grobstruktursimulationen mit adaptiven Gittern zu ermöglichen, war jedoch   
die Modifikation der Algorithmen zur Interpolation von turbulenter und
kinetischer Energie bei der Erzeugung von Gittern, damit die turbulente
Dissipation lokal auf verschiedenen Längenskalen erhalten bleibt. 
 
Nach der Lösung sehr vieler numerischer und technischer Probleme, die uns der
schlecht gewartete Enzo-Code leider aufzwang, ist es uns im Rahmen dieser
Arbeit gelungen zu zeigen, dass die Annahme lokaler Erhaltung der turbulenten 
Dissipation zu einem konsistenten Skalierungsverhalten der turbulenten Energie
in adaptiven Gittercodes führen kann. Wir konnten auch zeigen, dass unsere
Modifikation nicht zu einer Verletzung der Energieerhaltung führt.

Motiviert von diesen Ergebnissen verwendeten wir unser neues numerisches
Modell zur Simulation von Galaxienhaufen. Im Rahmen dieser Arbeit wurden dazu
eine hochaufgelöste Simulation mit und eine ohne Turbulenzmodell
durchgeführt, um den Einfluss unseres Turbulenzmodells auf die Entstehung eines
Galaxienhaufens zu untersuchen. Die Auswertung der Simulationen ergab folgendes
Bild:
\begin{itemize}
\item Unser Turbulenzmodell hat keinen signifikanten Einfluss auf Massenanteile
der unterschiedlichen Gasphasen im interstellaren Medium.
\item Die zeitlichen Entwicklung der turbulenten Energie lässt darauf 
schließen, dass alle Gasphasen  auf der Längenskala einer Galaxie genug Zeit
hatten, um im heutigen Universum eine turbulente Kaskade auszubilden. Wir
konnten auch zeigen, dass die Produktion und Dissipation turbulenter Energie 
im heutigen Universum praktisch im Gleichgewicht zu sein scheint.
\item Die Turbulenz auf der Längenskala einer Galaxie ist Unterschallturbulenz,
die mittlere turbulente Machzahl bei einer Rotverschiebung $z=0$ beträgt etwa
$0.2$.
\item Große Fluktuationen der turbulenten Energie bei beginnender
Galaxienhaufenbildung deuten darauf hin, dass \glqq gewaltsames\grqq
Verschmelzen
kleinerer zu großen Strukturen zu turbulenten Strömungen führt.
\item Turbulenz in Galaxienhaufen wird durch die Akkretion kleinerer
Strukturen nur am äußeren Rand ($r>R_{vir}$) getrieben. Dabei
lässt sich aus der räumlichen Verteilung der turbulenten Energie auf die
Akkretionsgeschichte des Galaxienhaufens zurückschließen, zumindest solange,
bis die turbulenten Geschwindigkeitsfluktuationen in thermische Energie
umgewandelt sind.
\item Aus dem Skalierungsverhalten der turbulenten Energie lässt sich ableiten,
dass Kräfte auf einer Längenskala von  $\approx \unit[100]{kpc\ h^{-1}}$
Energie in das System einkoppeln und sich zu kleineren Längenskalen hin eine
turbulente Kaskade ausbildet. In den Radialprofilen des Galaxienhaufens findet
man den Maximalwert der turbulenten Energie bei $r=0.5\ R_{vir}$,
wahrscheinlich verursacht durch die starke Abbremsung und damit verbundenen
Verwirbelung von Material, wenn es auf den Rand des Galaxienhaufen 
$r=R_{vir}$ trifft.  
\item Aus den Radialprofilen folgt auch, dass nur im Zentrum ein
signifikanter Einfluss unseres Turbulenzmodells auf die thermodynamischen
Eigenschaften des Galaxienhaufens besteht. Aus dem effektiven adiabatischen
Index ergibt sich, dass mit unserem Turbulenzmodell das interstellare Medium im
Zentrum des Galaxienhaufens kühlt, d.h. die Entropie und die Temperatur sind
niedriger, die Dichte und Geschwindigkeitsdispersion höher als ohne
Turbulenzmodell.
\end{itemize}

Ausgehend vom letztgenannten Punkt kann man auch vermuten, dass Turbulenz als
Mechanismus zur Lösung des \glqq Überkühlungsproblems\grqq\ nicht in Frage
kommt. Desweiteren führten vorläufige, niedrigaufgelöste Simulationen von
Galaxienhaufen zu dem interessanten Resultat, dass die Kolmogorovkonstante im
Skalierungsgesetz der turbulenten Geschwindigkeitsfluktuationen im
Galaxienhaufenzentrum eine
Größenordnung höher als in inkompressibler Turbulenz zu sein scheint.

Ein wichtiger Punkt bei der Quantifizierung von Turbulenz, der speziell in
unserer Arbeit offensichtlich wurde, ist die Skalenabhängigkeit der turbulenten
Energie und damit auch des turbulenten Druckes. Oft wird dies in der
astrophysikalische Literatur nicht berücksichtigt, was viele Aussagen
hinsichtlich der Stärke von Turbulenz unvollständig macht.
Die Frage jedoch, ob mit einer skalenabhängigen turbulenten Energie, wie sie
sich aus der Kolmogorov-Theorie und in unserem Turbulenzmodell ergibt,
wirklich Turbulenz in selbstgravitierenden Strömungen korrekt beschrieben
werden kann, können wir nicht beantworten. Ein besseres Verständnis
selbstgravitierender, turbulenter Strömungen scheint daher unerlässlich und
extrem wichtig für die weitere Erforschung von Turbulenz in der Astrophysik.

