\chapter{Dimensional analysis}\label{dimanal}
We want to write down the general equations of fluid dynamics in dimensionless
form. Therefore we introduce the following dimensionless quantities
\begin{align*}
r_i^*=\frac{r_i}{l_0} &\Rightarrow
\pd{r_i}=\pd{r_i^*}\ppd{r_i}{r_i^*}=\frac{1}{l_0}\pd{r_i^*},\\
v_i^*=\frac{v_i}{v_0},\\
t^*=\frac{t}{l_0} &\Rightarrow
\pd{t}=\pd{t^*}\ppd{t}{t^*}=\frac{1}{t_0}\pd{t^*},\\
\rho^*=\frac{\rho}{\rho_0}.
\end{align*}
Inserting these into the continuity equation \eqref{eq:mass} we get
\begin{align*}
\frac{\rho_0}{t_0}\pd{t^*}\rho^* 
+ \frac{v_0 \rho_0}{l_0} \pd{r_j^*}(v_j^*\rho^*) &= 0 \ | 
\cdot \frac{l_0}{\rho_0 v_0},\\
\underbrace{\frac{l_0}{v_0 t_0}}_{Sr}\pd{t^*}\rho^* 
+ \pd{r_j^*}(v_j^*\rho^*) = 0.
\end{align*}
This derivation shows that solutions of the continuity equation
are similar, if the Strouhal number $Sr = \frac{l_0}{v_0 t_0}$ is the
same. Flows with a Strouhal number $Sr=0$ are so called stationary flows.
Nevertheless, the Strouhal number is most often set to one, by assuming
$v_o=\frac{l_0}{t_0}$. 
Using the additional dimensionless quantities
$p^*=\frac{p}{p_0},\sigma_{ij}^*=\frac{\sigma_{ij}}{\sigma_0},
g^*=\frac{g}{g_0 }$
in the momentum equation \eqref{eq:mom} yields
\begin{align*}
\frac{\rho_0 v_0}{t_0}\pd{t^*}(\rho^* v_i^*) 
+ \frac{\rho_0 v_0^2}{l_0}\pd{r_j^*}(v_j^* \rho^* v_i^*) &= 
- \frac{p_0}{l_0}\pd{r_i^*}p^* 
+\frac{\sigma_0}{l_0}\pd{r_j^*}\sigma_{ij}^*
+\rho_0 g_0 \rho^* g_i^*\ | \cdot \frac{l_0}{\rho_0 v_0^2},\\
\underbrace{\frac{l_0}{v_0 t_0}}_{Sr}\pd{t^*}(\rho^* v_i^*) 
+ \pd{r_j^*}(v_j^* \rho^* v_i^*) &= 
-\underbrace{\frac{p_0}{\rho_0 v_0^2}}_{Ma^{-2}_{iso}}\pd{r_i^*}p^* 
+\underbrace{\frac{\sigma_0}{\rho_0 v_0^2}}_{Re^{-1}}\pd{r_j^*}\sigma_{ij}^*
+\underbrace{\frac{\rho_0 g_0 l_0}{\rho_0 v_0^2}}_{Fr^{-1}} \rho^* g_i^*.
\end{align*}
The occurring dimensionless numbers are the isothermal Mach number $Ma_{iso}$,
which is related to the Euler number $Eu$ or the Ruark number $Ru$ like
$Ma^2_{iso}=Eu=Ru^{-1}$, the Froude number $Fr$, which is related to the
Richardson number $Ri$ like $Fr=Ri^{-1}$ and the Reynolds number $Re$.
All these numbers measure the importance of the term they are related to
compared to the nonlinear advection term $\pd{r_j^*}(v_j^* \rho^* v_i^*)$,
e.g. for high Mach numbers, the pressure term $\pd{r_i^*}p^*$ becomes less and
less important compared to the advection term; for high Reynolds numbers the 
stress term $\pd{r_j^*}\sigma_{ij}^*$ becomes less and less important compared
to the advection term, and the equation shows more and more nonlinear behavior.
For a newtonian fluid with
\begin{align}
\sigma_{ij} \simeq \eta \ppd{r_j}{v_i} = 
\frac{\eta_0 v_0}{l_0} \eta^* \ppd{r_j^*}{v_i^*},
\end{align}
where we introduced the dimensionless quantity $\eta^*=\frac{\eta}{\eta_0}$,
we get $\sigma_0= \frac{\eta_0 v_0}{l_0}$ and therefore we can express the
Reynolds number\footnote{We neglected the second viscosity $\zeta$. In 
principle there exists a second Reynolds number $Re_2=\frac{\rho_0 l_0 v_0
}{\zeta_0}$.} like
\begin{align}
Re=\frac{l_0 \rho_0 v_0^2 }{\eta_0 v_0} = \frac{\rho_0 l_0 v_0 }{\eta_0}.
\end{align}
Playing the same game with the equation for the internal energy 
\begin{align}
\pd{t} \rho e_{int} + \pd{r_j} v_j \rho e_{int} = T \lra{\pd{t} \rho s +
\pd{r_j} v_j \rho s} -p \pd{r_j} v_j \label{eq:eint}
\end{align}
using $e_{int}^*=\frac{e_{int}}{u_0}$ we get
\begin{align*}
\underbrace{\frac{l_0}{v_0 t_0}}_{Sr} \pd{t^*}\rho^* e_{int}^* 
+ \pd{r_j^*} v_j^* \rho^* e_{int}^* &= 
-\underbrace{\frac{p_0}{\rho_0 u_0}}_{Ga_1} p^*\pd{r_j^*}v_j^*
+\underbrace{\frac{\sigma_0}{\rho_0 u_0}}_{Ga_2} \sigma_{ij}^* \pd{r_j^*}v_i^*.
\end{align*}
The new dimensionless quantities that occur in the energy equation seem to have
no name in the literature, but we will call them "Gamma1" (Ga1) and "Gamma2"
(Ga2) for now, since they are related to the adiabatic coefficient.
This can be seen by replacing $p_0 p^*$ according to equation
\begin{align}
p_0 p^* = (\gamma-1) \rho_0 u_0 \rho^* e_{int}^*,
\end{align}
which is valid for an an ideal, nonisothermal ($\gamma \neq 1$) gas. Doing this
we get\footnote{We cannot get rid of Ga2 in the same way, since therefore we
would have to assume an equation relating $\sigma_0 \sigma_{ij}^*$ to the
internal energy. But this would only be possible, if we would assume that the
internal energy is a tensorial quantity, which is not the way how
internal energy is defined normally.}
\begin{align*}
 Sr \cdot \pd{t^*}\rho^* e_{int}^* 
+ \pd{r_j^*} v_j^* \rho^* e_{int}^* &= 
-(\gamma-1) \rho^* e_{int}^* \pd{r_j^*}v_j^*
+Ga_2 \cdot \sigma_{ij}^* \pd{r_j^*}v_i^*.
\end{align*}
For a selfgravitating fluid we even have on more dimensionless quantity, which
appears, when we write down the dimesionless form of the Poisson equation of
gravity
\begin{align}
\underbrace{\frac{g_0}{4\pi G \rho_0 l_0}}_{C_G}\ppd{r_i^*}{g_i^*} = \rho^*.
\end{align}
But this quantity $C_G$ also seems to have no name in the literature
\citep[e.g.][]{Durst2007}.


\chapter{Properties of second order tensors}
A second order tensor can be decomposed into a symmetric and an antisymmetric
part in the following way
\begin{align}
T_{ij}=\underbrace{\frac{1}{2}\lra{T_{ij}+T_{ji}}}_{\text{symmetric}}
+\underbrace{\frac{1}{2}\lra{T_{ij}-T_{ji}}}_{\text{antisymmetric}}.
\end{align}
It can also be decomposed into an isotropic and deviatoric part by subtracting
and adding the trace of the tensor like
\begin{align}
T_{ij}=\underbrace{\frac{1}{n}\delta_{ij} T_{kk}}_{\text{isotropic}}
+\underbrace{T_{ij}-\frac{1}{n}\delta_{ij} T_{kk}}_{\text{deviatoric, 
tracefree}}.
\end{align}
Combining these two relations yields the general decomposition
\begin{align}
\begin{split}
T_{ij}=&
\overbrace{\frac{1}{n}\delta_{ij} T_{kk}}^{\text{isotropic}}
+
\overbrace{
\underbrace{\frac{1}{2}\lra{T_{ij}+T_{ji}-\frac{2}{n}\delta_{ij}
T_{kk}}}_{\text{symmetric, tracefree}}
+\frac{1}{2}\lra{T_{ij}-T_{ji}}}^{\text{deviatoric, tracefree}}\\[-1em]
&\underbrace{\hphantom{\frac{1}{n}\delta_{ij}
T_{kk}+\frac{1}{2}\lra{T_{ij}+T_{ji}-\frac{2}{n}\delta_{ij}
T_{kk}}}}_{\text{symmetric}}
\hphantom{+}
\underbrace{\hphantom{\frac{1}{2}\lra{T_{ij}-T_{ji}}}}_{\text{
antisymmetric}}.
\end{split}
\end{align}

An interesting relation can be found when computing the contraction of a
unsymmetric tensor $U_{ij} \neq U_{ji}$ with a  symmetric tensor
$V_{ij}=V_{ji}$
\begin{align}
U_{ij}V_{ij}
= \frac{1}{2} U_{ij}V_{ij} + \frac{1}{2} U_{ji}V_{ji} 
= \frac{1}{2} U_{ij}V_{ij} + \frac{1}{2} U_{ji}V_{ij}
= \frac{1}{2} \lra{U_{ij}+U_{ji}} V_{ij}.
\label{eq:uscontr}
\end{align}
In analogy one finds for the contraction of an unsymmetric tensor $U_{ij}$ with
an
antisymmetric tensor $W_{ij}=-W_{ji}$
\begin{align}
U_{ij}W_{ij} = \frac{1}{2} \lra{U_{ij}-U_{ji}} W_{ij}.
\label{eq:uascontr}
\end{align}

\chapter{Derivation of the stress tensor for a newtonian fluid}\label{stress}
We derive the stress tensor by considering the dissipation of a motionless
fluid seen by a rotating observer. This derivation is different from what
is found in the literature \citep[eg.][]{Greiner1991} and therefore presented
here.

It is generally assumed that friction between fluid elements is proportional
to the area of their surfaces. So in general the frictional or viscous force
on a fluid element can be expressed like
\begin{align}
F_{visc,i}= \oint_A \sigma'_{ij} n_j dA = \int_V \pd{r_j} \sigma'_{ij} dV.
\end{align}
This force leads to an irreversible rise of temperature in the fluid or an
irreversible
decrease of kinetic energy expressed by the equation for the
dissipation\footnote{See \citet{Landau1991}.}
\begin{align}
\pd{t} E_{kin,visc} = - \int_V \sigma'_{ij} \pd{r_j} v_i dV.
\end{align}
For a motionless fluid ($v_i=0$) and for a fluid with constant velocity 
($\ppd{r_j}{v_i} = 0$) this integral is zero. But also a rotating observer
of a motionless fluid should not see a rise in the temperature of a fluid
\footnote{We do not consider here the a rigidly rotating fluid as is often
done in the literature, because a rigidly rotating fluid is unphysical. This is
so, because a rigidly rotating fluid can never fulfill the
boundary condition $v_i=0$. However, for a rotating observer, the boundary
is also rotating, so that the boundary condition for a boundary at distance $R$
is
$v_i=\epsilon_{ijk}\omega_j R_k$ and there is no contradiction to the velocity 
field \eqref{eq:rotfield}.}
that means
\begin{align}
 \int_V \sigma'_{ij} \pd{r_j} v_i dV = 0 .\label{eq:nodiss}
\end{align}
A rotating observer of a motionless fluid sees a velocity field of the form
\begin{align}
v_i=\epsilon_{ijk}\omega_j r_k \label{eq:rotfield}.
\end{align}
where $\omega_j$ is the angular velocity vector and $r_k$ is the position
vector.
It can be shown that for such a velocity field the Jacobian is antisymmetric
\citep{Greiner1991}, that means
\begin{align}
\ppd{r_j}{v_i} = -\ppd{r_i}{v_j}.
\end{align}
Using this and equation \eqref{eq:uascontr} in equation \eqref{eq:nodiss} we get
\begin{align}
\int_V \frac{1}{2}\lra{\sigma'_{ij}-\sigma'_{ji}} \pd{r_j} v_i dV = 0.
\end{align}
This relation can only be fulfilled if the stress tensor $\sigma'_{ij}$
is symmetric
\begin{align}
\sigma'_{ij} = \sigma'_{ji}.
\end{align}
For a newtonian fluid is it assumed that the stress tensor is proportional only
to the first derivatives of the velocity field. Together with the requirement of
symmetry the most general form of such a tensor is
\begin{align}
\sigma'_{ij} = a\lra{\ppd{r_i}{v_j}+\ppd{r_j}{v_i}}+b \delta_{ij}\ppd{r_k}{v_k}.
\end{align}
Usually the trace is split off the first term and added to the second term so
\begin{align}
\sigma'_{ij} = a \lra{\ppd{r_i}{v_j}+\ppd{r_j}{v_i}
-\frac{2}{3}\delta_{ij}\ppd{r_k}{v_k}}+\lra{\frac{2
a}{3}+b}\delta_{ij}\ppd{r_k}{v_k}.
\end{align}
Using the definitions $a = \eta$ and $\frac{2 a}{3}+b=\zeta$ we get the form
most common in literature
\begin{align}
\sigma'_{ij} =  
2\eta\lrb{\frac{1}{2}\lra{\ppd{r_j}{v_i}+\ppd{r_i}{v_j}}
-\frac{1}{3}\delta_{ij}\ppd{r_k}{v_k}}
+\zeta \delta_{ij}\ppd{r_k}{v_k}.
\end{align}

\chapter{Fourier transform and structure functions}
The continuous one-dimensional Fourier transform in $k$-space $F(k)$ of some
function in
$x$-space $f(x)$is defined like
\begin{align}
F(k)&=\ft\iinf f(x) e^{-ikx} dx\ \ \text{(Fourier transform)}.
\end{align}
Using the Fourier transform on a function twice will produce
the original function again, but mirrored at the origin.
That's why one conventionally defines an inverse Fourier
transform\footnote{Nature does not know about the inverse Fourier transform.
If you have some optical device, which produces the Fourier transform
of some image and you use it twice on your image you will get a mirrored
image!},
that will generate the not mirrored original function again, when used on the
Fourier transform of a function
\begin{align}
f(x)&=\ft\iinf F(k) e^{ikx} dx\ \ \text{(inverse Fourier transform)}.
\end{align}
In three dimensions one defines the Fourier transform like
\begin{align}
F(\vec{k})=\ffft\iiiinf f(\vec{x}) e^{-i\vec{k}\vec{x}} dV, \\
f(\vec{x})=\ffft\iiiinf F(\vec{k}) e^{i\vec{k}\vec{x}} dK.
\end{align}
In cartesian coordinates the kernel of the Fourier transform 
$e^{-i\vec{k}\vec{x}}=e^{-i(k_xx+k_yy+k_zz)}$
separates and so the three-dimensional Fourier transform of a function
which separates in cartesian coordinates $f(\vec{x})=a(x)b(y)c(z)$ is 
also separable
\begin{align*}
F(\vec{k})=A(k_x)B(k_y)C(k_z)=
\ffft\iinf a(x) e^{-ik_xx} dx 
\iinf b(y) e^{-ik_yy} dy
\iinf c(z) e^{-ik_zz} dz.
\end{align*}
Thats why we like to use cartesian coordinates when we are using Fourier
transforms.
\section{Fourier transform of a delta function}
An important result can be derived by computing the inverse Fourier
transform of the Fourier transform of a delta function
\begin{align*}
\delta(x-x_0) &= \ft\iinf\ft\iinf \delta(x-x_0) e^{-ikx} dx e^{ikx} dk\\
&=\fft\iinf e^{-ikx_0} e^{ikx} dk = \fft\iinf e^{ik(x-x_0)} dk.
\end{align*}
From this we get that the inverse Fourier transform of a
constant is the delta function
\begin{align*}
\ft\iinf e^{ik(x-x_0)} dk = \sqrt{2\pi} \delta(x-x_0).
\end{align*}
Taking the complex conjugate of this equation and making use of the fact
that $\delta^*(x-x_0) = \delta(x-x_0)$ we get as a definition for the delta
function
\begin{align}
\delta(x-x_0)= \fft\iinf e^{\pm ik(x-x_0)} dk.
\end{align}
Using this we can derive the astonishing result
\begin{align}
\iinf f(x) dx = \sqrt{2\pi} F(0)
\end{align}
as can be seen from
\begin{align*}
\iinf f(x) dx &= \iinf \ft \iinf F(k) e^{ikx} dk dx 
= \ft \iinf F(k) \underbrace{\iinf e^{ikx}}_{2\pi \delta(k)} dx dk\\
&= \sqrt{2\pi} \iinf F(k) \delta(k) dk = \sqrt{2\pi} F(0).
\end{align*}

\section{Convolution theorem}
The Fourier transform of the product of two function in $k$-space is
\begin{align*}
&\ft\iinf F(k) G(k) e^{ikx} dk =\\
&= \ft \iinf \ft \iinf f(x') e^{-ikx'} dx' \ft\iinf g(x'') e^{-ikx''} dx''
e^{ikx} dk\\
&=\ffft \iiiinf f(x') e^{-ikx'} g(x'') e^{-ikx''} e^{ikx} dx' dx'' dk\\
&=\ffft \iiinf f(x') g(x'') 
\underbrace{\iinf e^{-ik(x'+x''-x)} dk}_{2\pi \delta(x''-(x-x'))} dx' dx'' \\
&=\ft\iinf f(x') \iinf g(x'') \delta(x''-(x-x')) dx'' dx' \\
&=\ft \iinf f(x') g(x-x') dx' = h(x).
\end{align*}
The integral $h(x)$ is called convolution of the functions $f(x)$ and $g(x)$.
So the convolution theorem says that
\begin{align}
h(x) = \ft \iinf f(x') g(x-x') dx' = \ft\iinf F(k) G(k) e^{ikx} dk.
\end{align}

\section{Autocorrelation and Wiener-Khinchin Theorem}
The autocorrelation of a function is defined as\footnote{Note that
with our definition of the Fourier transform we cannot define  
the autocorrelation function as $h_{AC}(x) = \ft \iinf f(x') f^*(x+x') dx'$,
because we could then not derive the Wiener-Khinchin theorem.}
\begin{align}
h_{AC}(x) = \ft \iinf f^*(x') f(x+x') dx'.
\end{align}
The Wiener-Khinchin Theorem states that 
\begin{align}
\ft \iinf f^*(x') f(x+x') dx' = \ft \iinf \abs{F(k)}^2 e^{ikx} dk,
\end{align}
which can be proved in analogy to the convolution theorem
\begin{align*}
&\ft \iinf f^*(x') f(x+x') dx' =\\
&=\ft \iinf f^*(x') \iinf f(x'') \delta(x''-(x+x')) dx'' dx'\\
&=\ft \iinf f^*(x') \iinf f(x'') \fft \iinf e^{-ik(x'+x''-x)} dk dx'' dx'\\
&=\ft \iinf \ft \iinf f^*(x') e^{ikx'} dx' \ft \iinf f(x'') e^{-ikx''} dx''
e^{ikx} dk\\
&=\ft \iinf F^*(k) F(k) e^{ikx} dk = \ft \iinf \abs{F(k)}^2 e^{ikx} dk.
\end{align*}
A special case of the Wiener-Khinchin theorem is Parseval's theorem
\begin{align}
\iinf \abs{f(x)}^2 dx = \iinf \abs{F(k)}^2 dk, 
\end{align}
which can be obtained from the Wiener-Khinchin theorem for $x=0$
\begin{align*}
h_{AC}(0)&= \ft \iinf f^*(x') f(x') dx' = \ft \iinf \abs{f(x)}^2 dx \\
&= \ft \iinf \abs{F(k)}^2 e^{ik0} dk = \ft \iinf \abs{F(k)}^2 dk.
\end{align*}

\section{Structure functions} \label{structfunc}
A structure function of order $p$ is defined as\footnote{See
\citet{Pope2000}.}
\begin{align}
S_p(f(x))=\fil{\lrb{f(x+x')-f(x')}^p} = \ft \iinf \lrb{f(x+x')-f(x')}^p
dx'.
\end{align}
The second order structure function is related to the spectrum
$\abs{F(k)}$ of the function $f$ like
\begin{align}
\ft \iinf \lrb{f(x+x')-f(x')}^2 dx' = \frac{2}{\sqrt{2\pi}}\iinf (1-e^{ikx})
\abs{F(k)}^2 dk, \label{eq:structspec}
\end{align}
which can be proved\footnote{A sketch of this prove can also be found in
\citet[Appendix G]{Pope2000}.} by expanding the second order structure function 
\begin{align*}
&S_2(f(x))=\ft \iinf \lrb{f(x+x')-f(x')}^2 dx'\\
&= \ft \lrb{\iinf \abs{f(x+x')}^2 dx' 
- 2 \iinf f^*(x')f(x+x')dx'
+ \iinf \abs{f(x')}^2 dx'}.
\end{align*}
Substituting $x''=x+x'$ in the first term we get
\begin{align*}
S_2(f(x)) &= \ft \lrb{\iinf \abs{f(x'')}^2 dx'' 
- 2\iinf f^*(x')f(x+x')dx'
+\iinf \abs{f(x')}^2 dx'}\\
&= \frac{2}{\sqrt{2\pi}} \lrb{\iinf \abs{f(x')}^2 dx'-\iinf f^*(x')f(x+x')dx'}.
\end{align*}
Using Parseval's and the Wiener-Khinchin theorem we obtain the final result
\begin{align*}
&S_2(f(x))=\frac{2}{\sqrt{2\pi}} \lrb{\iinf \abs{F(k)}^2 dk
-\iinf \abs{F(k)}^2 e^{ikx} dk} \\
&= \frac{2}{\sqrt{2\pi}}\iinf (1-e^{ikx})\abs{F(k)}^2 dk. 
\end{align*}

The structure functions used in the theory of Kolmogorov
are the so called longitudinal structure functions of the velocity, which are
defined as
\begin{align}
S_2(v_{\parallel}(l)) =  
\fil{\lra{\lrb{\vec{v}(\vec{x}+\vec{l})-\vec{v}(\vec{x})}\cdot
\frac{\vec{l}}{l}}^p} =
\fil{\lra{v_{\parallel}(\vec{x}+\vec{l})-v_{\parallel}(\vec{x})}^p}.
\end{align}
They are related to the longitudinal velocity
spectrum\footnote{In the literature this is often called
kinetic energy spectrum, but this is only true for
incompressible flows.} $\abs{V_{\parallel}(k)}^2$ via equation
\eqref{eq:structspec}.
Sometimes also second order transverse structure functions are measured.
These are defined as
\begin{align}
S_2(v_{\perp}(l)) =  
\fil{\lra{\frac{\abs{\lrb{\vec{v}(\vec{x}+\vec{l})-\vec{v}(\vec{x})}\times
\vec{l}}}{l}}^p}.
\end{align}
The behavior of the second order transverse structure functions for
homogeneous turbulence is uniquely determined by the longitudinal structure
function \citep[p. 192, Eqs. (6.28)]{Pope2000}. They also show the
characteristic $2/3$-slope as predicted for the longitudinal structure functions
\citep[p.60]{Frisch1995}.

In general structure functions of vectorial quantities like the velocity are
tensors, e.g. the general second order structure function of the velocity can
be defined as
\begin{align}
S_{ij}(\vec{x},\vec{l}) =  
\fil{\lrb{v_i(\vec{x}+\vec{l})-v_i(\vec{x})}
\lrb{v_j(\vec{x}+\vec{l})-v_j(\vec{x})}}.
\end{align}
But it can be shown that for local isotropy only the longitudinal
structure function $S_2(v_{\parallel}(l))=S_{11}$ and the transversal structure
$S_2(v_{\perp}(l))=S_{22}=S_{33}$ are unequal zero \citep{Pope2000}. Since 
the transverse structure function is determined by the longitudinal structure
function in case of local homogeneity, for homogeneous and isotropic 
turbulence $S_{ij}$ is determined by the single scalar
function $S_{11} = S_2(v_{\parallel}(l))$ \citep{Pope2000}.

The third order structure function used in Kolmogorov theory is
defined as 
\begin{align}
S_{111}(\vec{x},\vec{l}) = \fil{\lrb{v_1(\vec{x}+\vec{l})-v_1(\vec{x})}^3},
\end{align}
which is often simply called $S_3(v(l))$. So the famous four-fifths law of
Kolmogorov is actually true only for one component of the third order
structure function tensor, but again for homogeneous and isotropic turbulence
the third order structure function tensor $S_{ijk}$ is uniquely determined by
the single scalar function $S_{111}=S_3(v(l))$.

\chapter{The divergence equation}\label{diveq}
We start with the momentum equation \eqref{eq:mom}
\begin{align*}
\pd{t}(\rho v_i) + \pd{r_j}(v_j \rho v_i) &= -\pd{r_i}p + \pd{r_j}\sigma'_{ij}
-\rho \pd{r_i} \phi,
\end{align*}
where we assumed $g_i=-\pd{r_i} \phi$. If we make the substitutions 
$\pd{r_i} p \rightarrow \pd{r_j} p \delta_{ij}$ and 
$\pd{r_i} \phi \rightarrow \pd{r_j} \phi \delta_{ij}$ we can write it in the 
form
\begin{align*}
\pd{t}(\rho v_i) + \pd{r_j}(v_j \rho v_i + p \delta_{ij} - \sigma'_{ij}) 
= -\rho \pd{r_j} \phi \delta_{ij}.
\end{align*}
Taking the divergence of this equation we get
\begin{align*}
\pd{t}\lrb{\pd{r_i} (\rho v_i)} 
+ \frac{\partial^2}{\partial r_i \partial r_j}(v_j \rho v_i + p \delta_{ij} -
\sigma'_{ij}) 
= -\pd{r_i}\lra{\rho \pd{r_j} \phi \delta_{ij}},
\end{align*}
where we assumed that $\pd{t}$ and $\pd{r_i}$ commute. Using the continuity 
equation \eqref{eq:mass} we get a interesting form of the fluiddynamic 
equations
\begin{align}
\pdd{t}\rho 
- \frac{\partial^2}{\partial r_i \partial r_j}(v_j \rho v_i + p \delta_{ij} -
\sigma'_{ij}) 
= +\pd{r_i}\lra{\rho \pd{r_j} \phi \delta_{ij}}.\label{eq:divbeauty}
\end{align}
In case of no gravitation, the fluiddynamic equation can be written in a form
showing some similarity to a wave equation
\begin{align*}
\pdd{t}\rho 
- \frac{\partial^2}{\partial r_i \partial r_j}(v_j \rho v_i + p \delta_{ij} -
\sigma'_{ij}) 
= 0.
\end{align*}
But despite its simple form, this equation hides an extreme complexity.

Solving for pressure this equation is written like
\begin{align}
\pdd{r_i} p = \pdd{t}\rho- 
\frac{\partial^2}{\partial r_i \partial r_j}(\rho v_i
v_j-\sigma'_{ij})
\end{align}
and sometimes called the equation for the instantaneous pressure.

\chapter{Vlasov-Poisson Equations}\label{vlaspois}
The number of particles in the six-dimensional phase space element $dV dP$
at time $t$ can be expressed as
\begin{align}
N(t) = \int f(r_i,p_i,t) dV dP,
\end{align}
where $f(r_i,p_i,t)$ is the so called distribution function of particles in
phase space. The particle density can be expressed in terms of the distribution
function as
\begin{align}
\rho(r_i,t)=m n(r_i,t) = m \int f(r_i,p_i,t) dP,
\end{align}
where $n(r_i,t)$ is the number density of all particles and $m$ is the
particle mass. 

Without collisions, the distribution function satisfies the
equation
\begin{align}
\td{t} f(r_i,p_i,t) = 0.
\end{align}
Writing the time derivative explicitely we get the collisionless Boltzmann
equation
\begin{align}
\pd{t}f + \ppd{t}{r_i}\ppd{r_i}{f}+\ppd{t}{p_i}\ppd{f}{p_i} = 0.
\end{align}
With the velocity $v_i=\ppd{t}{r_i}$ and the gravitational force
$F_i=\ppd{t}{p_i}=m g_i$, this equation can be expressed as
\begin{align}
\pd{t}f + v_i \ppd{x_i}{f}+m g_i \ppd{f}{p_i} = 0. \label{eq:vlasov}
\end{align}
Together with the Poisson equation of gravity
\begin{align}
\pd{r_i} g_i = 4 \pi G m \int f(r_i,p_i,t) dP \label{eq:poisson}
\end{align}
equations \ref{eq:vlasov} and \ref{eq:poisson} are often called the
Vlasov-Poisson system of equations \citep{Peebles1980}. This system is used in
astrophysics to describe the evolution of collisionless matter interacting only
by gravity.

However in cosmological N-Body simulations it is not the Vlasov-Poisson system
of equations that is solved. In fact, one assumes that the solution of the
trajectories of $N$ particles determined by Newtons laws
\begin{align}
\ppd{t}{p_{i,j}} &= m_j g_{i,j},\\
g_{i,j} &= G \sum_{l=1}^N m_l \frac{r_{i,j}-r_{i,l}}{(r_{i,j}-r_{i,l})^3},
\end{align}
where $m_j,m_l$ and $r_{i,j},r_{i,l}$ are the position of the $j$th and
 $l$th particle respectively, can be 
interpreted as Monte-Carlo-Approximation of the Vlasov-Poisson system
\citep{Steinmetz1999}. So every particle in a cosmological N-Body simulation
can in fact represent a huge number of particles, which is a major
conceptual
difference to N-body simulations used to model planetary systems or stars in
star clusters, where each particle intends to mimic an actual physical body.

\chapter{Hydrostatic equilibrium}\label{stat}
\section{Standard derivation}
The equations of hydrostatic equilibrium can be obtained from the equations of
fluid dynamics \eqref{eq:mass}-\eqref{eq:etot} assuming $v_i =
0$\footnote{Actually this assumption is a little bit to stringent. For a
spherical symmetric system it is enough, that the average radial component of
velocity $\fil{v_r}=0$.}. This yields
\begin{align}
\pd{t}\rho &= 0,\\
\ppd{r_i}{p} &= \rho g_i. 
\end{align}
The gravitational force for a sphere with a mass profile $M(r)$ is 
\begin{align}
g_i=-\ppd{r_i}{\Phi} = -G \frac{M(r)}{r^2}, 
\end{align}
so the equation of hydrostatic equilibrium for such a configuration is
\begin{align}
\frac{1}{\rho}\ppd{r}{p} &=-G \frac{M(r)}{r^2}.\label{eq:static}
\end{align}
Substituting the ideal gas equation $p= \rho R_s T$ on the left hand side leads
to
\begin{align}
\frac{1}{\rho}\ppd{r}{p} = 
\frac{R_s}{\rho}\lra{\rho\ppd{r}{T}+T\ppd{r}{\rho}}=
R_s T \lra{\frac{1}{T}\ppd{r}{T}+\frac{1}{\rho}\ppd{r}{\rho}}=
R_s T \lra{\ppd{r}{\ln T}+\ppd{r}{\ln \rho}}.
\end{align}
Plugging the last result into the equation for hydrostatic equilibrium
\eqref{eq:static}, and solving for $M(r)$ gives an useful form of the
hydrostatic equilibrium equation
\begin{align}
M(r) &= - \frac{R_s T r}{G} \lra{r \ppd{r}{\ln T}+r \ppd{r}{\ln \rho}}
= - \frac{R_s T r}{G} \lra{\ppd{\ln r}{\ln T}+\ppd{\ln r}{\ln \rho}}.
\end{align}
\section{Derivation including turbulent pressure}
If we add a turbulent pressure $p_t$ to the ideal gas equation we get for the
total pressure
\begin{align}
p =p_{th}+p_t(l)=\rho R_s T+\frac{1}{3}\rho q^2(l).
\end{align}
If we substitute this into the equation for hydrostatic equilibrium
\eqref{eq:static}, we get an additional term due to the turbulent pressure
\begin{align}
\ppd{r}{p_t(l)} = \frac{1}{3}\rho q^2(l) 
\lra{\ppd{r}{\ln q^2(l)}+\ppd{r}{\ln \rho}}.
\end{align}
Therefore the total gravitational mass within the radius
$r$ assuming hydrostatic equilibrium including a turbulent pressure associated
with a length scale $l$ is
\begin{align}
M(r,l) &= - \frac{r}{G} 
\lrb{R_s T_g \lra{\ppd{\ln r}{\ln T_g}+\ppd{\ln r}{\ln \rho_g}}
+\frac{q^2(l)}{3}\lra{\ppd{\ln r}{\ln q^2(l)}+\ppd{\ln r}{\ln \rho}}}.
\end{align}

\chapter{Fluid dynamics in comoving coordinates}
\section{Introduction}
On large scales ($>$ 100Mpc) the distribution of matter 
in the universe is isotropic (it looks the same in all directions) 
and homogeneous (it is isotropic at each point). But only the space 
is assumed to be isotropic and homogenous. The observed expansion of 
the universe singles out a special direction in time.\footnote{In other
words: The universe is not a maximally symmetric 4-dimensional manifold, but can
be depicted as maximally symmetric 3-dimensional spacelike sheets 
in a 4-dimensional spacetime. The metric on such a manifold is the 
Robertson-Walker-metric.}

The physical distance on large scales\footnote{This is a very important 
point. If the space would also
expand on small scales we couldn't measure the expansion, because everything 
including our distance measurement device would expand. But on small scales
the universe is not homogenous. On small scales the metric of the universe is 
not a Robertson-Walker metric, but more like a Schwarzschild metric, which
is isotropic, but not homogenous.} 
between two points in such an expanding universe varies with time like
\begin{align}
r_i=a(t) x_i.
\end{align}
The factor $a$ is a dimensionless scale factor greater than zero, which must 
be the same for each component of the distance vector because of the assumed
isotropy.
The scale factor can only depend on the time $t$ and not on the position
$x_i$ because of the assumed homogeneity of space. 

The change of the distance with time in an expanding universe is then
\begin{align}
\dot{r}_i = \dot{a} x_i + a \dot{x}_i.  
\end{align}
The global velocity of a particle $v_i = \dot{r}_i$ which does not move 
relative to the expanding space ($\dot{x}_i = 0$) is
\begin{align}
\dot{v}_i = \dot{a} x_i = \fh r_i = H(t) r_i 
\end{align}
where $H$ is the so called Hubble parameter. Is a particle moving relative to
the expanding space ($\dot{x}_i \neq 0$) then we measure the additional 
local (also called proper) velocity $u_i = a \dot{x}_i$ of the particle. This
local
velocity can, according to special relativity, be never greater than the speed
of light
$c$. Nevertheless, the global velocity (e.g. the measured escape velocities of
galaxies at 
great distances) can be greater than $c$ \citep{Davis2004}. Generally the
physical velocity 
of a particle is the sum of global and local velocity
\begin{align}
v_i = \dot{a} x_i + u_i (x_i,t).
\end{align}

\section{Useful transformations}
From the definition of the distance $r_i$ and the velocity $v_i$ in comoving
coordinates we get 
\begin{align}
\pd{r_i}&=\fa\pd{x_i},\\
\pd{r_i}v_i&=\fa\pd{x_i}u_i+3\fh,\label{eq:cotrans4}\\
\pd{r_i}v_j&=\fa\pd{x_i}u_j + \fh\delta_{ij}, \label{eq:cotrans5}\\
\lra{\frac{\partial A}{\partial t}}_r+v_j \frac{\partial A}{\partial r_j} &= 
\lra{\frac{\partial A}{\partial t}}_x+\fa u_j \frac{\partial A}{\partial
x_j},\\
A(r_i,v_i,t) &\neq A(x_i,u_i,t).
\end{align}

With the help of transformation (\ref{eq:cotrans4}) and (\ref{eq:cotrans5}) we
can transform the stress tensor for a newtonian fluid in cartesian coordinates
\begin{align}
\sigma'_{ij}= 2\eta\lrb{\frac{1}{2}\lra{\pd{r_j}v_i+\pd{r_i}v_j}
-\frac{1}{n}\delta_{ij}\pd{r_k}v_k}+\zeta\delta_{ij}\pd{r_k}v_k
\end{align}
into the stress tensor for a newtonian fluid in comoving coordinates
\begin{align}
\sigma'_{ij}=& 
2\eta\lrb{\frac{1}{2} 
\lra{\fa\pd{x_j}u_i+\fh\delta_{ij}+\fa\pd{x_i}u_j+\fh\delta_{ji}} 
-\frac{1}{n}\delta_{ij}\lra{\fa\pd{x_k}u_k+n\fh}}\\
&+\zeta\delta_{ij}\lra{\fa\pd{x_k}u_k+n\fh}\\
=&2\eta\lrb{\frac{1}{2 a} 
\lra{\pd{x_j}u_i+\pd{x_i}u_j}+\fh\delta_{ij}-\frac{n}{n}\fh\delta_{ij}
-\frac{1}{n a}\delta_{ij}\pd{x_k}u_k}\\
&+\zeta\delta_{ij}\lra{\fa\pd{x_k}u_k+n\fh}\\
=&\fa\lrc{2\eta\lrb{\frac{1}{2} 
\lra{\pd{x_j}u_i+\pd{x_i}u_j}-\frac{1}{n}\delta_{ij}\pd{x_k}u_k}
+\zeta\delta_{ij}\lra{\pd{x_k}u_k+n\dot{a}}}.
\end{align}

\section{Equations in comoving coordinates}\label{UEcomoving}
With the help of the relations described in the last section, we can write
down the fluid dynamic equations in explicit comoving form
\begin{align}
\pd{t}\rho + \fa\pd{x_j}(u_j \rho) =& -3\fh\rho , \label{eq:commass}\\
\pd{t}(\rho u_i) + \fa\pd{x_j}(u_j \rho u_i) =& 
-\fa\pd{x_i}p + \fa\pd{x_j}\sigma'_{ij} +\rho g^*_i -4\fh \rho u_i,
\label{eq:commom}
\\
\begin{split}
\pd{t}(\rho e) + \fa\pd{x_j}(u_j \rho e) =& 
-\fa\pd{x_j}(u_j p) + \fa\pd{x_j}(u_i \sigma'_{ij}) + \fa u_i \rho g^*_i \\ 
&- 3 \fh(\rho e +\frac{1}{3}\rho u^2_i+p),\label{eq:cometot}
\end{split}
\end{align}
with Newtonian Gravity in comoving coordinates (Poisson Equation)
\begin{align}
\fa\pd{x_i}g^*_i=4\pi G,
\end{align}
where 
$g_i^*=-\fa\frac{\partial \varphi}{\partial x_i}$ and the gauge transformed
newtonian potential $\varphi=\phi+\frac{1}{2}a\ddot{a}x_i^2$.

The energy equation is the sum of the equation for the kinetic energy and the
internal energy
\begin{align}
\pd{t}(\rho e_k)+\fa\pd{x_j}(u_j \rho e_k) &= -\fa u_i \pd{x_i}p
+\fa u_i \pd{x_j}\sigma'_{ij}+\fa\rho u_i g^*_i-5\fh\rho e_k,\\
\pd{t}(\rho e_{int})+\fa\pd{x_j}(u_j \rho e_{int})&=
-\fa p \pd{x_j}u_j -\fa \sigma'_{ij}\pd{x_j}u_i- 3\fh(\rho e_{int} +p).
\end{align}

A even simpler form of the equations of fluid dynamics in comoving coordinates
can be found by expressing density and pressure in comoving coordinates.
The connection between the density in physical coordinates
$r_i=a(t) \cdot x_i$ and the comoving coordinates $x_i$ is given by
\begin{align}
\rho(r_i)=\frac{dM}{dV}=\frac{dM}{dr_1 dr_2 dr_3}
=\frac{1}{a(t)^3}\frac{dM}{dx_1 dx_2 dx_3}=\frac{1}{a(t)^3}
\rho(x_i)=\frac{1}{a(t)^3}\tilde{\rho}
\end{align}
and in analogy for the pressure
\begin{align}
p(r_i)=\frac{1}{a(t)^3} p(x_i)= \frac{1}{a(t)^3}\tilde{p}.
\end{align}
Because
\begin{align}
\pd{t}\rho = \pd{t}\frac{1}{a(t)^3}\tilde{\rho} =
\frac{1}{a(t)^3}\pd{t}\tilde{\rho} - 3 \fh\rho 
\end{align}
the source term on the right hand side of the momentum equation
\eqref{eq:commom} and energy conservation equation \eqref{eq:cometot} is reduced
and even vanishes in the mass conservation equation \eqref{eq:commass}, so that
we can write the system of equations for fluid dynamic in comoving coordinates
like
\begin{align}
\pd{t}\tilde{\rho} + \fa\pd{x_j}(u_j \tilde{\rho}) =& 0, \label{eq:commass2}\\
\pd{t}(\tilde{\rho} u_i) + \fa\pd{x_j}(u_j \tilde{\rho} u_i) =& 
-\fa\pd{x_i}\tilde{p}+ \fa\pd{x_j}\sigma'_{ij} 
+\tilde{\rho} g^*_i - \fh \tilde{\rho} u_i,
\label{eq:commom2}
\\
\begin{split}
\pd{t}(\tilde{\rho} e) + \fa\pd{x_j}(u_j \tilde{\rho} e) =& 
-\fa\pd{x_j}(u_j \tilde{p}) + \fa\pd{x_j}(u_i \sigma'_{ij}) 
+ \fa u_i \tilde{\rho} g^*_i \\ 
&- \fh(\tilde{\rho} e 
+\frac{1}{3}\tilde{\rho}u^2_i+\tilde{p}).\label{eq:cometot2}
\end{split}
\end{align}

\chapter{Color fields}\label{color}
There are two possibilities to implement a color field $c$ in a fluid code. One
can treat color like the density, i.e. the color variable obeys a conservation
law like the density
\begin{align}
\pd{t} c + \pd{r_j}(v_j c) &= 0.
\end{align}
In this case $c$ will exactly behave like density, if density and color have
the same initial value.

On the other hand one can treat it like a specific quantity obeying a
conservation law like
\begin{align}
\pd{t} \rho c + \pd{r_j}(v_j \rho c) &= 0.
\end{align}
In this case we see that if $c$ is spatially constant at a time $t_0$, i.e. 
$c(t_0)=c_0$, $\ppd{r_j}{c(t_0)}=0$, it will stay constant forever
\begin{align*}
\pd{t} \rho c + \pd{r_j}(v_j \rho c) &= 0 \\
\Leftrightarrow \ppd{t}{c} + v_j \ppd{r_j}{c} &= 0 \\
\Leftrightarrow \td{t} c &=0\\
\Rightarrow c &= const. = c_0
\end{align*}

