\chapter{Favre-filtered equations of fluid dynamics}\label{FE}
\section{Basic equations}
Using the Favre-Germano formalism developed in \ref{FGform} to filter the
equations of compressible selfgravitating fluid dynamics
\eqref{eq:mass}-\eqref{eq:etot} leads to

\begin{align}
\pd{t}\fil{\rho} + \pd{r_j}\hat{v}_j\fil{\rho} =&\ 0 \label{eq:film}\\
\pd{t}\fil{\rho}\hat{v}_i + \pd{r_j}\hat{v}_j\fil{\rho}\hat{v}_i =&
-\pd{r_i}\fil{p}+\pd{r_j}\fil{\sigma'_{ij}}+\fil{\rho}\hat{g}_i
-\pd{r_j}\hat{\tau}(v_i,v_j) \label{eq:filmom}
\\
\begin{split}
\pd{t}\fil{\rho}\hat{e}+\pd{r_j}\fil{\rho}\hat{v}_j\hat{e} =&
-\pd{r_j}\fil{v_j p}+\pd{r_j}\fil{v_i \sigma_{ij}}
+\fil{\rho}\hat{v}_i\hat{g}_i\\
&+\hat{\tau}(v_i,g_i)-\pd{r_j}\hat{\tau}(v_j,e)
\end{split}
\end{align}
with
\begin{align}
\hat{e}=\hat{e}_{int}+\frac{1}{2}\hat{v}_i\hat{v}_i
+\frac{1}{2}\frac{\hat{\tau}(v_i,v_i)}{\fil{\rho}}
\end{align}
and
\begin{align}
\hat{\tau}(v_j,e)=\hat{\tau}(v_j,e_{int})+\frac{1}{2}\hat{\tau}(v_j,v_i,v_i)
+\hat{v}_i\hat{\tau}(v_j,v_i)
\end{align}
Filtering the equation for the kinetic energy and the internal energy alone we
get:
\begin{align}
\begin{split}
\pd{t}\fil{\rho} \hat{e}_k+\pd{r_j}\hat{v}_j \fil{\rho} \hat{e}_k) =& 
-\fil{v_i \pd{r_i}p}+\fil{v_i \pd{r_j}\sigma'_{ij}}
+\fil{\rho} \hat{v}_i \hat{g}_i\\
&+\hat{\tau}(v_i,g_i)-\pd{r_j}\hat{\tau}(v_j,e_k) \label{eq:filkin}
\end{split}
\\
\begin{split}
\pd{t}\fil{\rho} \hat{e}_{int}+\pd{r_j}\hat{v}_j \fil{\rho} \hat{e}_{int} =&
-\fil{p \pd{r_j}v_j} + \fil{\sigma'_{ij}\pd{r_j}v_i}
-\pd{r_j}\hat{\tau}(v_j,e_{int})\label{eq:filint}
\end{split}
\end{align}
with
\begin{align}
\hat{e}_k=\frac{1}{2}\hat{v}_i\hat{v}_i
+\frac{1}{2}\frac{\hat{\tau}(v_i,v_i)}{\fil{\rho}}
\end{align}
and
\begin{align}
\hat{\tau}(v_j,e_k)=\frac{1}{2}\hat{\tau}(v_j,v_i,v_i)+\hat{v}_i\hat{\tau}(v_j,
v_i)
\end{align}
\section{Resolved energy and turbulent energy equations}
Multiplying the filtered equation for the momentum (\ref{eq:filmom}) with the
Favre-filtered velocity
$\hat{v}_i$ yields the balance equation for the resolved kinetic energy:
\begin{align}
\begin{split}
\pd{t}\fil{\rho}\frac{1}{2}\hat{v}_i\hat{v}_i 
+ \pd{r_j}\hat{v}_j\fil{\rho}\frac{1}{2}\hat{v}_i\hat{v}_i =&
-\hat{v}_i\pd{r_i}\fil{p}+\hat{v_i}\pd{r_j}\fil{\sigma'_{ij}}
+\fil{\rho}\hat{v}_i\hat{g}_i\\
&-\hat{v}_i\pd{r_j}\hat{\tau}(v_i,v_j) \label{eq:reskin}
\end{split}
\end{align}
Adding the equation for the resolved kinetic energy (\ref{eq:reskin}) to the
equation for the filtered internal energy (\ref{eq:filint}) one gets the
equation for the total resolved
energy $e_{res}=\hat{e}_{int}+\frac{1}{2}\hat{v_i}\hat{v_i}$:
\begin{align}
\begin{split}
\pd{t}\fil{\rho}e_{res}+\pd{r_j}\hat{v}_j\fil{\rho}e_{res}=&
-\hat{v}_i\pd{r_i}\fil{p}+\hat{v}_i\pd{r_j}\fil{\sigma'_{ij}}+\fil{\rho}\hat{v}
_i\hat{g}_i\\
&-\fil{p\pd{r_i}v_i}+\fil{\sigma'_{ij}\pd{r_j}v_i}
-\hat{v}_i\pd{r_j}\hat{\tau}(v_i,v_j)\\
&-\pd{r_j}\hat{\tau}(v_j,e_{int})\label{eq:resetot}
\end{split}
\end{align}
The arising four terms
$\fil{p\pd{r_i}v_i},\fil{\sigma'_{ij}\pd{r_j}v_i},
\hat{\tau}(v_i,v_j),\hat{\tau}(v_j,e_{int})$ in the total resolved energy
represent the coupling of the unresolved fluctuations to the filtered resolved
flow. We could now try to find equations based on quantities of the resolved
flow, to model each of these terms independent of each other.
Nevertheless we will see that the first three of these four terms can be
connected by an equation for another quantity, called the turbulent energy 
$\varepsilon_t$. From solving the equation for this quantity we get the three
terms $\fil{p\pd{r_i}v_i},\fil{\sigma'_{ij}\pd{r_j}v_i},\hat{\tau}(v_i,v_j)$.
Only the fourth term $\hat{\tau}(v_j,e_{int})$ is not connected with the
turbulent energy and has to be modeled separately. 

We get the balance equation for the turbulent energy\footnote{Interpreting the
quantity $\hat{\tau}(v_i,v_i) = \hat{\tau}_{ii}$ as an energy is only possible,
if $\hat{\tau}_{ii} \geq 0$. This is only guaranteed, if the filter 
convolution kernel is a semi-positive function in position
space \citep{Vreman1994,Sagaut2006}.} 
$\varepsilon_t=\fil{\rho} e_t =\frac{1}{2}\hat{\tau}(v_i,v_i)$ by
subtracting the balance equation for the resolved kinetic energy
(\ref{eq:reskin}) from the balance equation of the filtered kinetic energy
(\ref{eq:filkin}) :
\begin{align}
\begin{split}
\pd{t}\fil{\rho}e_t+\pd{r_j}\hat{v}_j \fil{\rho}e_t=&
-\lrb{\fil{v_i \pd{r_i}p}-\hat{v}_i\pd{r_i}\fil{p}} \\
&+\lrb{\fil{v_i \pd{r_j}\sigma'_{ij}}-\hat{v}_i\pd{r_j}\fil{\sigma'_{ij}}} \\
&+\hat{\tau}(v_i,g_i) 
-\frac{1}{2}\pd{r_j}\hat{\tau}(v_j,v_i,v_i) 
-\hat{\tau}(v_j,v_i)\pd{r_j}\hat{v_i}
\end{split}
\end{align}
For a better comparison with \citet{Schmidt2006} we will transform the following
terms like
\begin{align}
\fil{v_i \pd{r_i}p}&=\pd{r_i}\fil{v_i p}-\fil{p \pd{r_i} v_i}
\label{eq:trans1}\\
\hat{v}_i\pd{r_i}\fil{p}&=\pd{r_i}\hat{v}_i\fil{p}-\fil{p}\pd{r_i}\hat{v}_i\\
\fil{v_i \pd{r_j}\sigma'_{ij}}&=\pd{r_j}\fil{v_i \sigma'_{ij}}-\fil{\sigma'_{ij}
\pd{r_j}v_i}\\
\hat{v}_i\pd{r_j}\fil{\sigma'_{ij}}&=\pd{r_j}\hat{v}_i\fil{\sigma'_{ij}}-\fil{
\sigma'_{ij}}\pd{r_j}\hat{v}_i\label{eq:trans4}
\end{align}
and rewrite the balance equation for the turbulent energy:
\begin{align}
\begin{split}
\pd{t}\fil{\rho}e_t+\pd{r_j}\hat{v}_j \fil{\rho}e_t=&
-\pd{r_j}\lrb{\frac{1}{2}\hat{\tau}(v_j,v_i,v_i)+\fil{v_i p}-\hat{v}_i\fil{p}
-\fil{v_i \sigma'_{ij}}+\hat{v}_i\fil{\sigma'_{ij}}}\\
&-\lrb{\fil{p}\pd{r_i}\hat{v}_i-\fil{p \pd{x_i} v_i}} 
+\lrb{\fil{\sigma'_{ij}}\pd{r_j}\hat{v}_i-\fil{\sigma'_{ij} \pd{r_j}v_i}} \\
&+\hat{\tau}(v_i,g_i)-\hat{\tau}(v_j,v_i)\pd{x_j}\hat{v_i}
\end{split}
\end{align}
If we introduce now in analogy to \citet{Schmidt2006} the following
definitions
\begin{align}
-\mu &=\fil{v_i p}-\hat{v}_i\fil{p}\\
-\kappa &=\fil{v_i \sigma'_{ij}}-\hat{v}_i\fil{\sigma'_{ij}}\\
\mathbb{D} &= -\pd{r_j}\lrb{\frac{1}{2}\hat{\tau}(v_j,v_i,v_i) -\mu
+\kappa}\\
\fil{\rho} \lambda &=\lrb{\fil{p}\pd{r_i}\hat{v}_i-\fil{p \pd{r_i}
v_i}}\label{eq:def4}\\
\fil{\rho} \epsilon
&=-\lrb{\fil{\sigma'_{ij}}\pd{r_j}\hat{v}_i-\fil{\sigma'_{ij} \pd{r_j} v_i}}
\label{eq:def5}\\
\Gamma&=\hat{\tau}(v_i,g_i)
\end{align}
we can write the balance equation for the turbulence energy like
\begin{align}
\pd{t}\fil{\rho}e_t+\pd{r_j}\hat{v}_j
\fil{\rho}e_t=\mathbb{D}+\Gamma-\fil{\rho}(\lambda+\epsilon)
-\hat{\tau}(v_j,v_i)\pd{r_j}\hat{v_i}.\label{eq:et2}
\end{align}
With the help of equations (\ref{eq:trans1}) to (\ref{eq:trans4}) and the
definitions 
(\ref{eq:def4}) and (\ref{eq:def5}) we can also rewrite the equation
(\ref{eq:resetot}) for the total resolved energy:
\begin{align}
\begin{split}
\pd{t}\fil{\rho}e_{res}+\pd{r_j}\hat{v}_j\fil{\rho}e_{res}=&
-\pd{r_i}\hat{v}_i\fil{p}
+\pd{r_j}\hat{v}_i\fil{\sigma'_{ij}}+\fil{\rho}\hat{v}_i\hat{g}_i\\
&+\fil{\rho}(\lambda+\epsilon)
-\hat{v}_i\pd{r_j}\hat{\tau}(v_i,v_j)
-\pd{r_j}\hat{\tau}(v_j,e_{int}).\label{eq:resetot2}
\end{split}
\end{align}

\section{Summary}
The last two equations (\ref{eq:et2}) and (\ref{eq:resetot2}) together with
equation 
(\ref{eq:film}) and (\ref{eq:filmom}) (and additionally the Poisson equation for
the gravity term and the equation of state) form a complete system of partial
differential equations for fluid dynamics
\begin{align}
\pd{t}\fil{\rho} + \pd{r_j}\hat{v}_j\fil{\rho} =&\ 0, \label{eq:filmsum}\\
\pd{t}\fil{\rho}\hat{v}_i + \pd{r_j}\hat{v}_j\fil{\rho}\hat{v}_i =&
-\pd{r_i}\fil{p}+\pd{r_j}\fil{\sigma'_{ij}}+\fil{\rho}\hat{g}_i
-\pd{r_j}\hat{\tau}(v_i,v_j),\label{eq:filmomsum}\\
\begin{split}
\pd{t}\fil{\rho}e_{res}+\pd{r_j}\hat{v}_j\fil{\rho}e_{res}=&
-\pd{r_i}\hat{v}_i\fil{p}
+\pd{r_j}\hat{v}_i\fil{\sigma'_{ij}}+\fil{\rho}\hat{v}_i\hat{g}_i\\
&+\fil{\rho}(\lambda+\epsilon)
-\hat{v}_i\pd{r_j}\hat{\tau}(v_i,v_j)
-\pd{r_j}\hat{\tau}(v_j,e_{int}),
\end{split}\label{eq:filresetotsum}
\\
\pd{t}\fil{\rho}e_t+\pd{r_j}\hat{v}_j
\fil{\rho}e_t=&\ \mathbb{D}+\Gamma-\fil{\rho}(\lambda+\epsilon)
-\hat{\tau}(v_j,v_i)\pd{r_j}\hat{v_i}\label{eq:etsum},
\end{align}
where it is often useful to split the equation for resolved energy into an
equation for the resolved kinetic energy and internal energy respectively
\begin{align}
\begin{split}
\pd{t}\fil{\rho} \hat{e}_k+\pd{r_j}\hat{v}_j \fil{\rho} \hat{e}_k) =& 
-\hat{v}_i \pd{r_i}\fil{p}
+\hat{v}_i \pd{r_j}\fil{\sigma'_{ij}}
+\fil{\rho} \hat{v}_i \hat{g}_i\\
&-\hat{v}_i\pd{r_j}\hat{\tau}(v_i,v_j), \label{eq:filkinsum}
\end{split}
\\
\begin{split}
\pd{t}\fil{\rho} \hat{e}_{int}+\pd{r_j}\hat{v}_j \fil{\rho} \hat{e}_{int} =&
-\fil{p} \pd{r_j}\hat{v}_j 
+\fil{\sigma'_{ij}}\pd{r_j}\hat{v}_i
+\fil{\rho}(\lambda+\epsilon)\\
&-\pd{r_j}\hat{\tau}(v_j,e_{int}).\label{eq:filintsum}
\end{split}
\end{align}

The explicit forms of the quantities $\mathbb{D},\lambda,\epsilon,\Gamma$ and
$\hat{\tau}(v_i,v_j)$ are unknown and have to be modeled in terms of the
turbulence energy $e_t$.
The term $\hat{\tau}(v_j,e_{int})$ has to be modeled independently of $e_t$. The
models for all these terms represent our turbulence or subgrid model.

