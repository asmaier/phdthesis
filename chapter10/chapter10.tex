\chapter{Summary and Conclusions}\label{summ}
Turbulence is often invoked in astrophysics to explain phenomena, which are
not understood. However, studies that quantify the real impact of 
turbulence in astrophysical environments in general, or especially for the
formation of galaxy clusters, are not available. One reason for this is
obviously the lack of an accepted theory of compressible and/or supersonic
and/or
selfgravitating turbulence. A second reason is that the models used to
describe numerically the influence of turbulence (so-called 
large-eddy-simulations) are based on the notion of
filtering the fluid dynamic equations at a specific length scale, which is
incompatible with adaptive grid codes used to study astrophysical phenomena.

The aim of this work was to address the second problem, thereby developing,
implementing, and applying a new numerical scheme for modeling turbulent flows
over a great range of length scales suitable to treat astrophysical flows in
galaxy cluster cores or star forming regions. Because the cosmological fluid
code Enzo uses blockstructured adaptive mesh refinement in combination with
a low dissipative PPM-Solver, it was a natural choice to implement our ideas
into this code.

Still, great technical and numerical difficulties had to be circumvented.
Nevertheless, we could finally show that the
idea of our $\epsilon$-based approach to correct the velocity and energy fields
at grid refinement/derefinement according to local Kolmogorov scaling can
produce consistent results in simulations of driven turbulence. We demonstrated
that energy conservation and the scaling of turbulent energy in our adaptive
simulations is consistent with static grid simulations. 

Motivated by these results, we then attempted to use our new numerical scheme in
 simulations of galaxy cluster formation. Two high resolution runs of galaxy
cluster formation, one with and one without a turbulence model, have been
conducted to explore the influence of
turbulence modeled with our scheme on the formation of galaxy clusters. From
the analysis of these simulations, we conclude the following:
\begin{itemize}
\item Our turbulence model seems to have no significant influence on the mass
fractions of different gas phases of the ICM. 
\item The time development of turbulent energy in the simulation suggests that
basically all gas phases of the intracluster medium had enough time to develop
a turbulent cascade. In fact, we could show that our model seems to be near an
equilibrium of production and dissipation of turbulence, especially in the
cluster core. 
\item The turbulence at a length scale of galaxies 
($\approx \unit[10]{kpc\ h^{-1}}$) is subsonic, and the average
turbulent Mach number at these scale is found to be $0.2$ at redshift $z=0$. 
\item In the beginning of galaxy cluster formation great fluctuations of
turbulent energy can be seen, suggesting that violent merging can produce a
substantial amount of turbulence. 
\item Minor mergers can drive turbulence only at the outer rim 
($r>R_{vir}$) of the galaxy cluster. The spatial distribution of turbulent
energy traces the local merging history of a galaxy cluster until the turbulent
motions are dissipated into heat completely. 
\item From the scaling properties of turbulent energy it seems that energy
is injected at a scale of $\approx \unit[100]{kpc\ h^{-1}}$ cascading down to
smaller scales. From the radial profile of our cluster we found a peak of
turbulent energy at $r=0.5\ R_{vir}$, probably produced by the infall and 
strong deceleration of material, when it hits the virial boundary of the
cluster.
\item From the radial profiles of several thermodynamical quantities of the
galaxy cluster it is evident, that only inside the core ($r<0.1\ R_{vir}$)
can one find a significant influence of our turbulence model. The radial profile
of
the effective adiabatic index shows that the influence of the turbulent model
can be described as a kind of cooling, leading to lower entropy, lower
temperature, and therefore higher gas density and higher velocity dispersion in
the core. "Cooling" due to turbulence does not lead to an overcooling
problem, but it is not strong enough to explain the cool cores of galaxy
clusters.
\end{itemize}

The last result begs the question of how turbulence would influence a
simulation of cluster formation including cooling. If there is no nontrivial
interaction between cooling and the turbulence model, our results indicate that
turbulence would even enhance the overcooling problem in the core. So
suggesting turbulence as a heating mechanism that prevents galaxy cluster
cores from overcooling seems to be problematic. Nevertheless more
simulations of galaxy clusters, including different physics have to
be carried out to confirm our results. 

Interestingly preliminary results from low resolution simulations suggest, that
turbulent velocity in the cluster core obeys a Kolmogorov scaling law with
a Kolmogorov constant more than 10 times higher than in incompressible
simulations. Whether this finding is only a feature of our SGS model
or a universal feature of turbulence in the cluster core should be
investigated in the future.

More attention should also be given to the fact that turbulent energy and
thus unresolved turbulent velocity fluctuations are scale dependent. It is
often claimed in the astrophysical literature that the amount of
turbulence is a certain fraction of thermal energy or kinetic energy, without
specifying the length scale for which this statement was made. In the spirit of
Kolmogorov theory of turbulence, such statements are incomplete. This is
especially apparent in our adaptive grid simulations, where different grid
length scales at the same time are used to describe a flow. However, we have to
note that the idea of scale dependent velocity fluctuations poses
difficult conceptual problems. For example, the mass inside a certain radius $r$
from the equation of hydrostatic equilibrium is, including turbulent pressure,
also scale dependent\footnote{For a derivation see appendix \ref{stat}.}
\begin{align}
M(r,l) &= - \frac{r}{G} 
\lrb{R_s T_g \lra{\ppd{\ln r}{\ln T_g}+\ppd{\ln r}{\ln \rho_g}}
+\frac{q^2(l)}{3}\lra{\ppd{\ln r}{\ln q^2(l)}+\ppd{\ln r}{\ln \rho}}},
\end{align}
a fact, which is often overlooked. Arguing that turbulent pressure might
explain deviation from the mass found by estimates based on the hydrostatic
equilibrium, is therefore not advisable.

Nevertheless, it might also come out, that the ideas of Kolmogorov and scale
dependent velocity fluctuations are not useful in an astrophysical context.
Within
our work, we only showed how the influence of turbulence obeying basically
Kolmogorov scaling can be numerically treated and what kind of results can be
expected. We could not prove that turbulence in an astrophysical environment
really can be described in this way. Theoretically, Kolmogorov derived
his celebrated result assuming a forcing of turbulence restricted to the largest
length scales, so that in the limit of infinite Reynolds numbers an undisturbed
cascade down to smaller scales can develop. However, gravity is a force acting
on
all length scales, in contradiction to the ideas of Kolmogorov and our
turbulence model. A better understanding of selfgravitating turbulence is
therefore extremely important for the future of turbulence research in
general. 
That's why in the future comparisons between direct numerical simulations of
selfgravitating gas and simulations with our subgrid model should be conducted.
If these simulations support our SGS model (which would also show, that
Kolmogorov scaling is more universal than theory suggests), we can be
confident in saying, that with our FEARLESS ansatz we developed a unique tool
for describing turbulence which aside from cluster physics will lead to many
other applications in astrophysics.  

